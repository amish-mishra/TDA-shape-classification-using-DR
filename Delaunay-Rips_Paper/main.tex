% Write this paper as if it is my dissertation and then if I want to publish it, then prune it to be a succint paper

\documentclass[letterpaper,titlepage]{article}
\usepackage{Amish_Style}
\newcommand{\incfig}[1]{%
    \def\svgwidth{\columnwidth}
    \import{./figures/}{4pt_filtration.pdf_tex}
}

\addbibresource{bibliography.bib} %Imports bibliography file

\title{Delaunay-Rips Paper}
\author{Amish Mishra and Francis Motta}
\date{\today}


\begin{document}
\maketitle

\section{Introduction}
Welcome to our paper!

\section{Background}

\subsection{Topological Data Analysis}
    We are living in an information age where data-driven decision making is a huge area of interest. With so much data at our hands, many questions naturally arise.
    \begin{itemize}
        \item How do we extract relevant information from the data? \item How do we even know what is relevant and what is not? 
        \item If we are unable to visualize large quantities of data, especially data in high dimensions, then how do we know what sort of data set we are inspecting? 
        \item Further, how can we compare information extracted from one data set with another?
    \end{itemize}
    These are the sorts of difficult and fascinating questions tackled in the field of Topological Data Analysis (TDA). From the name itself, TDA hints at leveraging ideas borrowed from topology with data analysis techniques to measure and quantify qualitative features of data. At a more nuanced level, TDA appears as the child of Algebraic Topology, Computer Science, Statistics, Data Analysis, and Computational Geometry. Results from each field have found beautiful applications in TDA and have shed new light on applicability of theoretical results from mathematics.
    
    
\subsection{Simplicial Complexes}
    The idea behind extracting topological features from a given point-cloud requires there to be a method of assigning some sort of ``shape" to the data. Only then are we able to study its associated topological properties. The scenario we face is that we have a finite dimensional metric space out of the point-cloud that we need to assign a shape to. With the human eye, we may be able to make out some appropriate surface our data set can live on. However, this situation pokes at a fundamental question that arises when computing: how can we get a computer to do what a human can do? We need to introduce the idea of a \textit{simplicial complex}, a computational efficient way for a computer to build a surface onto a data set.
    
    \begin{definition}
        A simplicial complex is a collection $K$ of non-empty subsets of a set $K_0$ such that $\{v\} \in K$ for all $v \in K_0$, and $\tau \subset \sigma$ and $\sigma \in K$ guarantees that $\tau \in K$. The elements of $K_0$ are called vertices of $K$, and the elements of $K$ are called simplices. Additionally, we say that a simplex has dimension $p$ or is a $p$-simplex if it has cardinality of $p+1$. We use $K_p$ to denote the collection of $p$-simplices. The $k$-skeleton of $K$ is the union of the sets $K_p$ for all $p \in \{0,1,\dots,k\}$. If $\tau$ and $\sigma$ are simplices such that $\tau \subset \sigma$, then we call $\tau$ a face of $\sigma$, and we say that $\tau$ is a face of $\sigma$ of codimension $k'$ if the dimensions of $\tau$ and $\sigma$ differ by $k'$. The dimension of $K$ is defined as the maximum of the dimensions of its simplices. A map of simplicial complexes, $f: K \to L$, is a map $f: K_0 \to L_0$ such that $f(\sigma) \in L$ for all $\sigma \in K$. \cite{edelsbrunner}
    \end{definition}
    
\subsection{Simplicial Homology}

\subsection{Vietoris-Rips Complex} \label{rips}
One of the simplest ways to build a complex on a data set $X$ is by considering the pairwise distance between the points. The approach described here is an algorithmic, bottom-up approach that adds higher and higher dimensional simplices to the complex for a fixed scale. For a given scale $\varepsilon>0$, if $d(x,x')\leq 2\varepsilon$ for $x,x' \in X$, then we add the edge between $x$ and $x'$ into our complex. Once all of the edges are added, we add the higher dimensional simplices if their faces are already in the complex. That is, we add the $k$-simplex $\sigma = \{x_0,x_1,\dots,x_k\}$ to the complex if every subset $u \subset \sigma$ is already in the complex. Formally, we define the Vietoris-Rips complex \cite{roadmap} for scale $\varepsilon>0$ 
$$VR_{\varepsilon}(X) = \{\sigma \subseteq X\ |\ d(x,x') \leq 2\varepsilon,\ \forall x,x' \in \sigma\}.$$

\subsection{Delaunay Triangulation}
Although the Vietoris-Rips complex is simple to implement, constructing it on data sets with large numbers of points results in computation drawback. As the scale increases, we see that adding certain simplices does not affect the homology of the point cloud. We need some way to ``weed" out these extraneous simplices as we construct our complex to increase computational efficiency. Turning to a tool of Computational Geometry, we incorporate the Delaunay Triangulation in our construction. Our definition is adapted from ``A roadmap for the computation of persistent homology"\cite{roadmap}. Assume our data $X$ lives in the space $\R^n$. Let $x \in X$. We define
$$V_x = \{p \in \R^d\ |\ d(p,x) \leq d(p, x')\ \forall x' \in X\}.$$
Each $V_x$ is called a Vornoi cell. Note that $\{V_x\}_{x \in X}$ forms a cover of $\R^n$. This cover is known as the Vornoi decomposition of $\R^n$ with respect to $X$. To construct the Delaunay triangulation from this cover, we connect $x,x' \in X$ with an edge if $V_x$ and $V_{x'}$ are neighbors (that is, the Vornoi cells share a wall). When the points in $X$ are in general position, this gives us a graph (1-skeleton) on $X$ that is known as the Delaunay Triangulation. Formally, we define \cite{edelsbrunner}
$$Del(X) = \{\sigma \subset X\ |\ \bigcap_{u \in \sigma} V_u \neq \emptyset\}.$$
In this paper, we will be only be using the edges of the Delaunay Triangulation. We call it the Delaunay 1-skeleton and define it as
$$Del_{1}(X) = \{\sigma \in Del(X)\ |\ dim(\sigma) = 1\}.$$
We will use $Del_1(X)$ as the underlying graph structure when defining the Delaunay-Rips complex in section \ref{del-rips:def}.

\subsection{Persistence}


\section{Delaunay-Rips Complex}
\subsection{Definition and Construction} \label{del-rips:def}
The Delaunay-Rips complex is our new method of building a complex on a data set $X$. It utilizes the conceptual simplicity of the Vietoris-Rips complex while cutting down on the number of high dimensional and extraneous simplices. This computational speed-up is by virtue of using the Delaunay Triangulation as the ``backbone" of building the Vietoris-Rips complex on $X$. The idea is that we build the Vietoris-Rips complex on $X$ but only add edges if the edges occur in the Delaunay 1-skeleton of the point cloud. The higher dimensional $k$-simplices are then added the traditional way they are in section \ref{rips}. Formally, we define the Delaunay-Rips Complex for a given scale $\varepsilon>0$
$$DR_{\varepsilon}(X) = \{\sigma \subseteq X\ |\ d(x,x') \leq 2\varepsilon,\ \forall x,x' \in \sigma \text{ and } \sigma \in Del_1(X) \}.$$


\subsection{Example Data Set}
\begin{enumerate}
    \item Demonstrate construction on a small data-set (5-8 point data-set).
\end{enumerate}

\subsection{Run-time Analysis Comparison}
\begin{enumerate}
    \item How does this scale as dimensions are increased?
    \item How does this scale as points are added?
\end{enumerate}


\subsection{Persistence Diagram Instability}
The Delaunay-Rips construction gains computational efficiency at the cost of stability. We demonstrate a simple, yet clear example of how this instability can arise.

Let ($\mathcal{P}$, $d_{GH}$) be the space of point clouds equipped with the Gromhov-Hausdorff metric and let ($\mathcal{D}$, $W_\infty$) be the space of Persistence Diagrams equipped with the bottle neck metric. Define
$$\varphi: \mathcal{P} \to \mathcal{D}$$
$$\varphi(P):= Pers(P)$$
where $Pers(P)$ is the persistence diagram of the point cloud $P$ constructed using the Delaunay-Rips complex. Our example comes from 4 points taken in $\R^2$ where the instability is demonstrated as the discontinuity of $\varphi$.

Let $P \in \mathcal{P}$ as $P = \{(-1,0),(\frac{1}{2},\frac{\sqrt{3}}{2}),(\frac{1}{2},-\frac{\sqrt{3}}{2}),(1,0)\}.$ Note that the points all lie on the unit circle, so the Delaunay 1-skeleton has an edge between every pair of points (See figure). Thus, $\varphi(P)$ has no $H_1$ class with non-zero persistence using the Delaunay-Rips filtration, as can be verified by the reader.

Fix $\varepsilon=0.1$. We now show that for any $\delta > 0$, there exists $P' \in \mathcal{P}$ such that $d_{GH}(P,P')< \delta$, but $W_\infty(\varphi(P), \varphi(P')) \geq \varepsilon.$ Take $P' = \{(-1,0),(\frac{1}{2},\frac{\sqrt{3}}{2}),(\frac{1}{2},-\frac{\sqrt{3}}{2}),(1-x,0)\}$ with $0<x < \delta < \frac{2-\sqrt{3}}{2}$. This is a small perturbation of $P$ by pushing the point $(1,0)$ inside the unit circle thereby putting the points in general position. We only work with $\delta<2-\sqrt{3}$ so that $\varphi(P')$ maintains an $H_1$ class with non-zero persistence; for this example to work, we further need $\delta < \frac{2-\sqrt{3}}{2}$. We just compute the Hausdorff distance $d_H$ between $P$ and $P'$ in the plane taking the isometric embedding of $P$ to be the map that sends each of its points to itself in $\R^2$ and the same embedding for $P'$. Since the Gromov-Hausdorff distance is the infimum of $d_H(f(P),g(P'))$ over all isometric embeddings $f:P \to X$ and $g: P' \to X$ into any metric space $X$, $d_H(P,P')$ serves as an upper bound for $d_{GH}(P,P').$ We find that
$$d_{GH}(P,P')\leq d_H(P,P')=x<\delta.$$
Recall that $\varphi(P)$ has no $H_1$ class with non-zero persistence. Thus, to compute $W_\infty(\varphi(P),\varphi(P'))$, we must match the $H_1$ class of $\varphi(P')$ with the diagonal. The $H_1$ class of $\varphi(P')$ has birth $\sqrt{3}$ and death $2-x$ as calculated in the Appendix, section \ref{boundary_mat}. Using the max norm, we find
$$d: = W_\infty(\varphi(P),\varphi(P')) = 2-x-\sqrt{3} \geq 2-\frac{2-\sqrt{3}}{2} -\sqrt{3} \geq 0.1 = \varepsilon.$$
Hence, our map $\varphi$ is discontinuous at $P$. This gives us insight into when the Delaunay-Rips construction of the Persistence Diagram experiences instability--namely when points are not in general position. We now have motivation to ask if we have stability of the PD when the underlying Delauany-Rips complex does not change under a perturbation of the point cloud.


\subsection{Stability in a Neighborhood}
\begin{enumerate}
    \item What is the best our method can do? Use knowledge on stability of Delaunay Triangulation.
    Let $P$ denote our point cloud and $P'$ denote the perturbed point cloud. We use $|xy|$ notation to denote the Euclidean distance between points $x,y \in \R^n$. Let $V_x$ denote the closed Voronoi region for the point $x \in P$. Here is a lemma that should work in $\R^n$:
    \begin{lem}\label{paperlemma}
    Two points $p_i,p_j \in P$ are strong Voronoi neighbors if and only if there exists an $m$ such that
    $$\max\{|mp_i|,|mp_j|\} < \min_{k \neq i,j}|mp_k|.$$
    \end{lem}
    
    We claim
    \begin{thm}
    Let $p_i,p_j \in P$ be strong Voronoi neighbors and $m \in V_{p_i} \cap V_{p_j}$. For $\varepsilon = \min\{\frac{1}{4}(\min_{k \neq i,j}|mp_k|-|mp_j|), \frac{1}{2}\min_{i \neq j} |p_i p_j|\}$, an $\varepsilon$-perturbation of $P$ leaves strong Voronoi neighbors as strong Voronoi neighbors.
    \end{thm}
    \begin{proof}
        Let $p_i,p_j \in P$ satisfy Lemma \ref{paperlemma} with $m \in V_{p_i} \cap V_{p_j}$. Note that $0<\varepsilon<\frac{1}{2}\min_{i \neq j} |p_i p_j|$ to ensure a unique correspondence between the points of $P$ and the points of $P'$. We begin with the conclusion of Lemma \ref{paperlemma}:
        $$\max\{|mp_i|,|mp_j|\} < \min_{k \neq i,j}|mp_k|$$
        $$2\max\{|mp_i|,|mp_j|\} < 2\min_{k \neq i,j}|mp_k|$$
        $$\min_{k \neq i,j}|mp_k|+3\max\{|mp_i|,|mp_j|\} < 3\min_{k \neq i,j}|mp_k|+\max\{|mp_i|,|mp_j|\}$$
        $$\frac{1}{4}\min_{k \neq i,j}|mp_k| +\frac{3}{4}\max\{|mp_i|,|mp_j|\} < \frac{3}{4}\min_{k \neq i,j}|mp_k| + \frac{1}{4}\max\{|mp_i|,|mp_j|\}$$
        % $$\max\{|mp_i|,|mp_j|\}+\frac{1}{4}\min_{k \neq i,j}|mp_k|-\frac{1}{4}\max\{|mp_i|,|mp_j|\} < \min_{k \neq i,j}|mp_k| -\frac{1}{4}\min_{k \neq i,j}|mp_k|+\frac{1}{4}\max\{|mp_i|,|mp_j|\}$$
        $$\max\{|mp_i|,|mp_j|\}+\frac{1}{4}(\min_{k \neq i,j}|mp_k|-\max\{|mp_i|,|mp_j|\}) < \min_{k \neq i,j}|mp_k| -\frac{1}{4}(\min_{k \neq i,j}|mp_k|-\max\{|mp_i|,|mp_j|\})$$
        \begin{equation} \label{sandwich_eq}
        \begin{aligned}
            \max\{|mp_i|,|mp_j|\}+\varepsilon \leq \max\{|mp_i|,|mp_j|\}+\frac{1}{4}(\min_{k \neq i,j}|mp_k|-\max\{|mp_i|,|mp_j|\}) \\ < \min_{k \neq i,j}|mp_k| -\frac{1}{4}(\min_{k \neq i,j}|mp_k|-\max\{|mp_i|,|mp_j|\})\leq \min_{k \neq i,j}|mp_k| - \varepsilon.
        \end{aligned}
        \end{equation}
        
        Now, without loss of generality, let
        $$|mp_i'| = \max\{|mp_i'|,|mp_j'|\}.$$
        We note by the triangle inequality that
        \begin{equation}\label{lower_side}
            \max\{|mp_i'|,|mp_j'|\} \leq |mp_i| + |p_ip_i'| \leq \max\{|mp_i|,|mp_j|\} + \varepsilon.
        \end{equation}
        Similarly, we have by the triangle inequality
        \begin{equation}\label{upper_side}
            \min_{k\neq i,j}|mp_k| - \varepsilon = \min_{k\neq i,j}(|mp_k| - \varepsilon) \leq \min_{k\neq i,j}(|mp_k|-|p_kp_k'|) \leq \min_{k \neq i,j} |mp_k'|.
        \end{equation}
        Putting together equations \ref{sandwich_eq}, \ref{lower_side}, \ref{upper_side}, we have
        $$\max\{|mp_i'|,|mp_j'|\} \leq \max\{|mp_i|,|mp_j|\}+\varepsilon < \min_{k \neq i,j}|mp_k| - \varepsilon \leq \min_{k \neq i,j} |mp_k'|.$$
        Thus, we now apply Lemma \ref{paperlemma} and have that $p_i'$ and $p_j'$ remain strong Voronoi neighbors.
    \end{proof}
%%%% Proof with the max and mins written out
    % \begin{proof}
    %     Let $p_i,p_j \in P$ satisfy Lemma \ref{paperlemma} with $m \in V_{p_i} \cap V_{p_j}$. Note that $0<\varepsilon<\frac{1}{2}\min_{i \neq j} |p_i p_j|$ to ensure a unique correspondence between the points of $P$ and the points of $P'$. We define the following to make reading the proof easier
    %     $$|mp| := \max\{|mp_i|,|mp_j|\}$$
    %     $$|mp_k|:= \min_{k \neq i,j}|mp_k|$$
    %     We begin with the conclusion of Lemma \ref{paperlemma}:
    %     $$\max\{|mp_i|,|mp_j|\} < \min_{k \neq i,j}|mp_k|$$
    %     $$2\max\{|mp_i|,|mp_j|\} < 2\min_{k \neq i,j}|mp_k|$$
    %     $$\min_{k \neq i,j}|mp_k|+3\max\{|mp_i|,|mp_j|\} < 3\min_{k \neq i,j}|mp_k|+\max\{|mp_i|,|mp_j|\}$$
    %     $$\frac{1}{4}\min_{k \neq i,j}|mp_k| +\frac{3}{4}\max\{|mp_i|,|mp_j|\} < \frac{3}{4}\min_{k \neq i,j}|mp_k| + \frac{1}{4}\max\{|mp_i|,|mp_j|\}$$
    %     $$\max\{|mp_i|,|mp_j|\}+\frac{1}{4}\min_{k \neq i,j}|mp_k|-\frac{1}{4}\max\{|mp_i|,|mp_j|\} < \min_{k \neq i,j}|mp_k| -\frac{1}{4}\min_{k \neq i,j}|mp_k|+\frac{1}{4}\max\{|mp_i|,|mp_j|\}$$
    %     $$\max\{|mp_i|,|mp_j|\}+\frac{1}{4}(\min_{k \neq i,j}|mp_k|-\max\{|mp_i|,|mp_j|\}) < \min_{k \neq i,j}|mp_k| -\frac{1}{4}(\min_{k \neq i,j}|mp_k|-\max\{|mp_i|,|mp_j|\})$$
    %     \begin{equation} \label{sandwich_eq}
    %         \max\{|mp_i|,|mp_j|\}+\varepsilon \leq \max\{|mp_i|,|mp_j|\}+\frac{1}{4}(\min_{k \neq i,j}|mp_k|-\max\{|mp_i|,|mp_j|\}) < \min_{k \neq i,j}|mp_k| -\frac{1}{4}(\min_{k \neq i,j}|mp_k|-\max\{|mp_i|,|mp_j|\})\leq \min_{k \neq i,j}|mp_k| - \varepsilon.
    %     \end{equation}
    %     Now, without loss of generality, let
    %     $$|mp_i'| = \max\{|mp_i'|,|mp_j'|\}.$$
    %     We note by the triangle inequality that
    %     \begin{equation}\label{lower_side}
    %         \max\{|mp_i'|,|mp_j'|\} \leq |mp_i| + |p_ip_i'| \leq \max\{|mp_i|,|mp_j|\} + \varepsilon.
    %     \end{equation}
    %     Similarly, we have by the triangle inequality
    %     \begin{equation}\label{upper_side}
    %         \min_{k\neq i,j}|mp_k| - \varepsilon = \min_{k\neq i,j}|mp_k - \varepsilon| \leq \min_{k\neq i,j}(|mp_k|-|p_kp_k'|) \leq \min_{k \neq i,j} |mp_k'|.
    %     \end{equation}
    %     Putting together equations \ref{sandwich_eq}, \ref{lower_side}, \ref{upper_side}, we have the theorem as desired.
    %     $$\max\{|mp_i'|,|mp_j'|\} \leq \max\{|mp_i|,|mp_j|\}+\varepsilon < |mp_k| - \varepsilon \leq \min_{k \neq i,j} |mp_k'|.$$
    % \end{proof}
\end{enumerate}

\section{Application of Delaunay-Rips}
\begin{enumerate}
    \item Demonstrate value by talking about as dimensions change and number of points change.
    \item Particular examples of how using special data sets affect the run-time of Rips/Alpha drastically but maybe not Del-Rips.
    \item Performance: accuracy in ML algorithm, or classification. Instability may cause performance to go down even though run-time is unaffected.
\end{enumerate}

\subsection{Synthetic Data}

\subsection{Real Data}



\section{Conclusion}



\subsection{Further Questions}

\section{Appendix}

\subsection{Boundary Matrix Calculation for Instability}\label{boundary_mat}

\begin{figure}[ht]
    \centering
    \incfig{Filtration}
    \caption{filtration}
    \label{fig:4pt_filtration}
\end{figure}

We have $P' = \{(-1,0),(\frac{1}{2},\frac{\sqrt{3}}{2}),(\frac{1}{2},-\frac{\sqrt{3}}{2}),(1-x,0)\}$ with $0< x < \delta < 2-\sqrt{3}$.
Our filtration has 4 key scale values, $t = 0<\sqrt{1-x+x^2}< \sqrt{3}< 2-\delta$ as shown in the figure. We construct our boundary matrix and reduce it using the standard algorithm:
$$B=\bordermatrix{ 
    &   a & b & c & d & bd & cd & ab & ac & ad & abd & acd\cr
    a & 0 & 0 & 0 & 0 &  0 &  0 &  1 &  1 &  1 &   0 &   0 \cr
    b & 0 & 0 & 0 & 0 &  1 &  0 &  1 &  0 &  0 &   0 &   0\cr
    c & 0 & 0 & 0 & 0 &  0 &  1 &  0 &  1 &  0 &   0 &   0\cr
    d & 0 & 0 & 0 & 0 &  1 &  1 &  0 &  0 &  1 &   0 &   0\cr
   bd & 0 & 0 & 0 & 0 &  0 &  0 &  0 &  0 &  0 &   1 &   0\cr
   cd & 0 & 0 & 0 & 0 &  0 &  0 &  0 &  0 &  0 &   0 &   1\cr
   ab & 0 & 0 & 0 & 0 &  0 &  0 &  0 &  0 &  0 &   1 &   0\cr
   ac & 0 & 0 & 0 & 0 &  0 &  0 &  0 &  0 &  0 &   0 &   1\cr
   ad & 0 & 0 & 0 & 0 &  0 &  0 &  0 &  0 &  0 &   1 &   1\cr
  abd & 0 & 0 & 0 & 0 &  0 &  0 &  0 &  0 &  0 &   0 &   0\cr
  acd & 0 & 0 & 0 & 0 &  0 &  0 &  0 &  0 &  0 &   0 &   0\cr
    } \qquad$$
$$\overline{B}=\bordermatrix{ 
    &   a & b & c & d & bd & cd & ab & ac & ad & abd & acd\cr
    a & 0 & 0 & 0 & 0 &  0 &  0 &  1 &  0 &  0 &   0 &   0 \cr
    b & 0 & 0 & 0 & 0 &  1 &  1 &  1 &  0 &  0 &   0 &   0\cr
    c & 0 & 0 & 0 & 0 &  0 &  1 &  0 &  0 &  0 &   0 &   0\cr
    d & 0 & 0 & 0 & 0 &  1 &  0 &  0 &  0 &  0 &   0 &   0\cr
   bd & 0 & 0 & 0 & 0 &  0 &  0 &  0 &  0 &  0 &   1 &   1\cr
   cd & 0 & 0 & 0 & 0 &  0 &  0 &  0 &  0 &  0 &   0 &   1\cr
   ab & 0 & 0 & 0 & 0 &  0 &  0 &  0 &  0 &  0 &   1 &   1\cr
   ac & 0 & 0 & 0 & 0 &  0 &  0 &  0 &  0 &  0 &   0 &   1\cr
   ad & 0 & 0 & 0 & 0 &  0 &  0 &  0 &  0 &  0 &   1 &   0\cr
  abd & 0 & 0 & 0 & 0 &  0 &  0 &  0 &  0 &  0 &   0 &   0\cr
  acd & 0 & 0 & 0 & 0 &  0 &  0 &  0 &  0 &  0 &   0 &   0\cr
    }. \qquad$$
The persistence pairs for the $H_0$ class with their persistence diagram coordinate (birth/death) come out as follows:
$$(a,N/A): (0,\infty)$$
$$(b,ab): (0,\sqrt{3})$$
$$(c, cd): (0,\sqrt{1-x+x^2})$$
$$(d, bd): (0,\sqrt{1-x+x^2}).$$
The $H_1$ classes come out as
$$(ad, abd): (2-x, 2-x)$$
$$(ac, acd): (\sqrt{3}, 2-x).$$
The only point that appears in the diagram off of the diagonal is $(\sqrt{3}, 2-x).$


\subsection{Pseudo-code Implementation}

\subsection{Github Repo of Actual, Clean Code}
\begin{enumerate}
    \item We want to compare the best implementation of Del-Rips with Ripser and Cechmate's Alpha.
\end{enumerate}

\subsection{Machine Specs}
\begin{enumerate}
    \item Eluktronics laptop
\end{enumerate}


\section{Bibliography}

\printbibliography[
heading=bibintoc,
title={Whole bibliography}
] %Prints the entire bibliography with the title "Whole bibliography"

%Filters bibliography
% \printbibliography[heading=subbibintoc,type=article,title={Articles only}]
% \printbibliography[type=book,title={Books only}]

% \printbibliography[keyword={physics},title={Physics-related only}]
% \printbibliography[keyword={latex},title={\LaTeX-related only}]
\end{document}